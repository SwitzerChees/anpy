\subsection{Rekursionstiefe und Fibonacci-Folge}
Der folgende Abschnitt beschäftigt sich mit der Rekursionstiefe
der zuvor beschriebenen Implementation. Das bedeutet die
folgende Implementation ermöglicht es herauszufinden wieviele
gestapelte rekursive Aufrufe die naive Implementation machen
muss um eine bestimmte Fibonacci Zahl zu berechnen.
\begin{mdframed}[backgroundcolor=bg]
    \inputminted{Python}{src/count_fibonacci.py}
\end{mdframed}
Beim Aufruf der Funktion kommen folgende Outputs zustande:
\begin{mdframed}[backgroundcolor=bg]
    \inputminted{Python}{src/count_fibonacci_test.py}
\end{mdframed}
Was auf den ersten Blick auffällt ist das geringe Wachstum 
der rekursiven Aufrufe im Vergleich zu den hohen Fibonacci-Folgen
welche daraus generiert werden kann. Die obige Implementation
zur Ermittlung der Rekursiontiefe gehört ungefähr folgender 
Gruppe der Laufzeitkomplexität:
\begin{equation}
    \mathcal{O}(n*n^2 + 1)
\end{equation}