\section{Naive Implementierung}
Im folgenden beschäftigen wir uns mit einer naiven
Implementierung der Fibonacci-Folge mit Python.
Diese soll wie auch die mathematische Definition rekursiv geschrieben werden.
Als Editor wird VS Code mit Python Plugin verwendet.
Die Schwierigkeit der Implementierung besteht in den
ersten 2 Folgen da diese noch nicht der üblichen Logik folgen.
Daher ist für diese zwei Werte eine Grenzfallbehandlung
zu Implementieren. Dies wurde mithilfe einer \textit{if-else}
Anweisung in der Rückgabeanweisung umgesetzt:
\begin{mdframed}[backgroundcolor=bg]
    \inputminted{Python}{src/naive_fibonacci.py}
\end{mdframed}
Beim Aufruf der Funktion kommen folgende Outputs zustande:
\begin{mdframed}[backgroundcolor=bg]
    \inputminted{Python}{src/naive_fibonacci_test.py}
\end{mdframed}
Durch die oben durchgeführten \textit{print} Tests konnte
zudem die maximale Rekursionstiefe, die Python standardmässig
gesetzt hat, ermittelt werden. 
Diese liegt bei 1000. 
Wir werden diese Implementierung nun
auf verschiedene Arten analysieren um mehr über das 
Verhalten der Funktion beim Aufruf zu erfahren.

