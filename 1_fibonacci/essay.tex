\documentclass[12pt]{article}
\usepackage[a4paper, total={6in, 8in}, margin=1.1in]{geometry}
\usepackage[default,scale=0.95]{opensans}
\usepackage[section]{minted}
\usepackage{mdframed}
\usepackage{lingmacros}
\usepackage{tree-dvips}
\usepackage{listings}
\usepackage{blindtext}
\usepackage{amsmath}


\renewcommand\seriesdefault{l}
\renewcommand\mddefault{l}
\renewcommand\bfdefault{sb}
\renewcommand{\contentsname}{Inhalt}


\title{Semesterarbeit Teil 1: Rekursion mit Python anhand der Fibonacci-Folge}
\date{Februar, 2020}
\author{Patrick Michel\\
FFHS - Fernfachhochschule Schweiz\\
6. Semester AnPy\thanks{AnPy, Analysis mit Python, BSc INF 2017 Pas, BE1-I, FS20, Geuss Markus}
}

\begin{document}

\definecolor{bg}{rgb}{0.95,0.95,0.95}

\begin{titlepage}

\maketitle
\tableofcontents

\end{titlepage}

\section{Einleitung}
Die folgende Arbeit befasst sich mit der Fibonacci-Folge. 
Dabei handelt es sich um eine unendliche Reihe von natürlichen Zahlen.
Die Folge ist benannt nach Leonardo Fibonacci, 
der damit im 1300 Jahrhunder das Wachstum einer Kaninchenpopulation zu beschreiben versuchte.
Heute wird die Zahlenfolge in einer abgewandelter Form zum Beispiel auch
in der agilen Welt zur Schätzung von Arbeitsaufwänden verwendet (Story Points).
Die Zahlenfolge beginnt üblicherweise mit einer 0.
Darauf folgend zwei mal die 1 und danach wird die Folge nachvollziehbar.
Nachfolgend zur Nachvollziehbarkeit die definition der ersten 5 Folgen:
\begin{equation}
  f_0 := 0,
  f_1 := 1,
  f_2 := 1,
  f_3 := 2,
  f_4 := 3
\end{equation}
Daraus ergibt sich folgende allgemeine Definition:
\begin{equation}
  f_n := f_{n-1} + f_{n-2} \text{ für } n \ge 2
\end{equation}
Auf Basis dieser Definition sollen nun folgende Aufgaben
erledigt werden:
\begin{enumerate}
  \item Implementieren Sie eine Python-Funktion \textit{fib(n)}, die die n-te Fibonacci-Zahl bestimmt.
  \item Eine naive Implementierung setzt die obige Rekursionsgleichung direkt um. Schreiben Sie eine weitere Python-Funktion, die berechnet, wie viele Funktionsaufrufe on fib notwendig sind, um die n-te Fibonacci-Zahl zu berechnen.
  \item Vergleichen Sie die Anzahl der Funktionsaufrufe von fib zur Bestimmung einer Fibonacci-Zahl mit den Fibonacci-Zahlen selber. Können Sie eine Vermutung aufstellen?
  \item Verwenden Sie die Funktion \textit{time()} aus dem Modul time, um zu bestimmen, wie lange die Funktion fib benötigt, um eine Fibonacci-Zahl zu bestimmen.
  \item Implementieren Sie eine weitere Python-Funktion zur Berechnung der n-ten Fibonacci-Zahl, die möglichst effizient ist. (Hinweis: das kann rekursiv oder iterativ gelöst werden.)
\end{enumerate}
\subsection{Abstract}

\newpage

\section{Naive Implementierung}
Im folgenden beschäftigen wir uns mit einer naiven
Implementierung der Fibonacci-Folge mit Python.
Diese soll wie auch die mathematische Definition rekursiv geschrieben werden.
\subsection{Der Algorithmus}
Als Editor wird VS Code mit dem Python Plugin verwendet.
Die Schwierigkeit der Implementierung besteht in den
ersten 2 Folgen da diese noch nicht der üblichen Logik folgen.
Daher ist für diese zwei Werte eine Grenzfallbehandlung
zu Implementieren. Dies wurde mithilfe einer \textit{if-else}
Anweisung in der Rückgabeanweisung umgesetzt:
\begin{mdframed}[backgroundcolor=bg]
    \inputminted{Python}{src/naive_fibonacci.py}
\end{mdframed}
Beim Aufruf der Funktion kommen folgende Outputs zustande:
\begin{mdframed}[backgroundcolor=bg]
    \inputminted{Python}{src/naive_fibonacci_test.py}
\end{mdframed}
Durch die oben durchgeführten \textit{print} Tests konnte
zudem die maximale Rekursionstiefe, die Python standardmässig
gesetzt hat, ermittelt werden. 
Diese liegt bei 1000. 
Weiter kann durch die Analyse des Codes folgende
Laufzeit des Algorithmus definiert werden:
\begin{equation}
    \mathcal{O}(n + 1)
\end{equation}



\newpage

\section{Rekursionstiefe analysieren}
Der folgende Abschnitt beschäftigt sich mit der Rekursionstiefe
der zuvor beschriebenen Implementation. 
Im folgenden beschäftigen wir uns mit einer naiven
Implementierung der Fibonacci-Folge mit Python.
Diese soll wie auch die mathematische Definition rekursiv geschrieben werden.
Als Editor wird VS Code mit Python Plugin verwendet.
Die Schwierigkeit der Implementierung besteht in den
ersten 2 Folgen da diese noch nicht der üblichen Logik folgen.
Daher ist für diese zwei Werte eine Grenzfallbehandlung
zu Implementieren. Dies wurde mithilfe einer \textit{if-else}
Anweisung in der Rückgabeanweisung umgesetzt:
\begin{mdframed}[backgroundcolor=bg]
    \inputminted{Python}{src/count_fibonacci.py}
\end{mdframed}
Beim Aufruf der Funktion kommen folgende Outputs zustande:
\begin{mdframed}[backgroundcolor=bg]
    \inputminted{Python}{src/count_fibonacci_test.py}
\end{mdframed}
Durch die oben durchgeführten \textit{print} Tests konnte
zudem die maximale Rekursionstiefe, die Python standardmässig
gesetzt hat, ermittelt werden. 
Diese liegt bei 1000. 
Wir werden diese Implementierung nun
auf verschiedene Arten analysieren um mehr über das 
Verhalten der Funktion beim Aufruf zu erfahren.



\newpage

\subsection{Rekursionstiefe und Fibonacci-Folge im Vergleich}
Der folgende Abschnitt beschäftigt sich mit der Rekursionstiefe
der zuvor beschriebenen Implementation. Das bedeutet die
folgende Implementation ermöglicht es herauszufinden wieviele
gestapelte rekursive Aufrufe die naive Implementation machen
muss um eine bestimmte Fibonacci Zahl zu berechnen.
\begin{mdframed}[backgroundcolor=bg]
    \inputminted{Python}{src/count_fibonacci.py}
\end{mdframed}
Beim Aufruf der Funktion kommen folgende Outputs zustande:
\begin{mdframed}[backgroundcolor=bg]
    \inputminted{Python}{src/count_fibonacci_test.py}
\end{mdframed}
Was auf den ersten Blick auffällt ist das geringe Wachstum 
der rekursiven Aufrufe im Vergleich zu den hohen Fibonacci-Folgen
welche daraus generiert werden kann. Die obige Implementation
zur Ermittlung der Rekursiontiefe gehört zu folgender 
Gruppe der Laufzeitkomplexität:
\begin{equation}
    \mathcal{O}(n^2 + 1)
\end{equation}

\newpage

\section{Ergebnisse}
\blindtext

\section{Diskussion / Interpretation}
\blindtext

\section{Quellen}
\blindtext

\end{document}